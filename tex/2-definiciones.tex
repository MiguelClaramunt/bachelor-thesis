% \section{Definiciones}
\label{sec:defs}

Como hemos introducido en el capítulo \hyperref[chapter:1]{1}, han habido varios intentos de definir y compartimentar el concepto de \emph{fake news}, ya que de esta manera podemos estudiar de forma íntegra su repercusión. Consideramos interesante incluir estos análisis ya que permite tener una idea de la línea de pensamiento en la literatura, facilitando la comprensión de este fenómeno. Además, ayuda a delimitar el problema, pudiendo trabajar de forma concisa y efectiva, ya que como se detallará a continuación, es un término que estos últimos años ha sido protagonista en los debates de la academia. 

% Debido a que los trabajos que hemos utilizado se encuentran escritos en inglés, 

% A continuación, procederemos a analizar varias definiciones y tipificaciones del término.

Comenzamos con la definición de \citet{AlcottGentzkow2017}, que es con la cual se ha introducido este trabajo:

\phantomsection
\label{frag1esp}
\begin{quotation}
     ``artículos de noticias que son intencionada y verificablemente falsos, y que pueden inducir a equívoco a los lectores'' --- \hyperref[frag1eng]{Fragmento original en inglés}
\end{quotation}

La definición de \citet{Lazer2018} también sigue una línea de pensamiento similar, matizando algunos términos:

\phantomsection
\label{frag2esp}
\begin{quotation}
    ``información fabricada que simula el contenido de los medios de comunicación en la forma, pero no en el proceso organizativo o la intencionalidad''  --- \hyperref[frag2eng]{Fragmento original en inglés}
    --- 
\end{quotation}

Estas dos definiciones consiguen dibujar el aspecto de las \emph{fake news}, que consiguen hacerse pasar por artículos periodísticos verídicos y que se caracterizan por tener una intencionalidad de confundir a la población. Además, debido a que no se busca la veracidad, tampoco es necesario seguir los mismos principios organizativos que sigue el periodismo.

El análisis de \citet{Tandoc2017} permite arrojar más luz sobre la intencionalidad, la cual es llegar a que la población legitime tanto estos fragmentos de información como sus autores, alcanzando la misma legitimidad de los medios de comunicación:

\phantomsection
\label{frag3esp}
\begin{quotation}
    ``las \emph{fake news} se aproximan al aspecto y la esencia de las noticias reales, desde el aspecto de los sitios web hasta la redacción de los artículos o la inclusión de atribuciones en las fotografías. Las \emph{fake news} se esconden bajo un manto de legitimidad, adquiriendo cierta credibilidad al intentar aparentar noticias reales. Incluso, yendo más allá de la simple apariencia de una noticia, mediante el uso de \emph{bots}, las \emph{fake news} imitan la omnipresencia de las noticias construyendo una red de sitios falsos.''  --- \hyperref[frag3eng]{Fragmento original en inglés}
\end{quotation}

Aún así, estas definiciones son insuficientes, ya que según un estudio de \citet{Mourao2019}, se encontraron entre la gran mayoría de los fragmentos analizados una mezcla de información falsa, sensacionalismo, contenido sesgado y \emph{clickbait}\footnote{Entendemos \textit{clickbait} como el diseño de contenidos con el objetivo de llamar la atención de los lectores y atraerlos para que hagan clic en un enlace
a un sitio web mediante tácticas como historias sensacionalistas, titulares llamativos e imágenes, que funcionan
como cebo \citep{Chen2015,Blom2015}}. A esto se suma el hecho de que no todo el contenido falso que se difunde tiene estructura o apariencia de noticia. Por tanto, es crucial reformular el concepto, ya que no nos permitiría afrontar o incluso entender el problema en su totalidad.

Si consideramos todas las formas de información falsa en Internet como noticias falsas cuando estas no se presentan en formato de noticia, estamos atentando contra el rigor que se espera de una investigación calidad \citep{MacKenzie2011,Suddaby2010,Zhang2016}. Por otro lado, excluir estas formas de falsedad por no tener carácter de noticia podría mermar su relevancia al pasar por alto cuestiones del mundo real, como la negación del cambio climático y el llamado movimiento anti-vacunas, que se nutren de investigaciones, informes o publicidad fraudulentos o cuestionables, y de opiniones partidistas \citep{Khan2021}. Esto incluso puede desencadenar en errores de `tipo III'\ en la formulación de problemas de investigación: el hecho de centrarse en la cuestión inmediata sin tener en cuenta ``un problema más general y arquetípico'', impediría hacer una ``contribución académica más amplia y duradera a escala del problema genérico'' \citep{Rai2017}.

Es por ello que \citet{Khan2021} formula el problema utilizando como referencia el concepto de información de \citet{Mingers2018}: `el contenido proposicional de signos'.

\phantomsection
\label{frag4esp}
\begin{quotation}
    ``La información es un contenido proposicional puesto que propone la existencia de un estado concreto del mundo, `lo que debe ocurrir en el mundo para que el signo exista como y cuando existe'.'' --- \hyperref[frag4eng]{Fragmento original en inglés}
\end{quotation}

Y a partir del concepto de información formula los términos \emph{misinformation}, \emph{disinformation} y \emph{malinformation}:

\phantomsection
\label{frag5esp}
\begin{quotation}
    ``\underline{Misinformation.} Contenido proposicional de signos que tergiversa el estado del mundo sin intención de engañar [...]. Un área [...] en la (este fenómeno) es bastante común es el asesoramiento sanitario en comunidades en línea \citep{Venkatesan2014}, donde muchas personas difunden información falsa de forma no intencionada \citep{Myers2009}.'' \citep{Khan2021} --- \hyperref[frag5eng]{Fragmento original en inglés} \\

\phantomsection
\label{frag6esp}

    ``\underline{Disinformation.} Contenido proposicional de signos que tergiversa el estado del mundo con la intención de engañar.'' \citep{Khan2021} --- \hyperref[frag6eng]{Fragmento original en inglés} \\

\phantomsection
\label{frag7esp}

    ``\underline{Malinformation.} Contenido proposicional de signos que representa verazmente el estado del mundo con la intención de engañar [...] A menudo se asume que el engaño aparece en forma o como resultado de la mentira descarada y otras formas de \emph{disinformation}. Sin embargo, el engaño puede producirse igualmente en forma o como resultado de una sutil manipulación de la información que no necesariamente tergiversa el mundo pero que tiene la intención de engañar \citep{McCornack2009,McCornack2014,Wardle2018a}. Algunos ejemplos son las medias verdades y los montajes, que se refieren a información incompleta o selectiva proporcionada con la intención de engañar \citep{Fallis2016}.'' \citep{Khan2021} --- \hyperref[frag7eng]{Fragmento original en inglés}
\end{quotation}

Habiendo estudiado exhaustivamente qué constituyen las \emph{fake news} y hasta dónde abarcan, podemos establecer concretamente el ámbito de aplicación de nuestro estudio. Para ello, tomaremos la definición de \emph{\underline{disinformation}}: ``contenido proposicional de signos que tergiversa el estado del mundo con la intención de engañar.''

Debido a que la tipología del problema es compleja, limitaremos nuestra área de aplicación a noticias, ya que estas son las más fáciles de recopilar gracias a repositorios de páginas web catalogadas como \emph{fake news}. Asimismo, estas son fácilmente categorizables como verdaderas o falsas, facilitando el entrenamiento de los modelos, que se hará de manera supervisada.